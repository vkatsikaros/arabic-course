\section*{Κούφια Ρήματα}
Είναι τα ρήματα που στο \ar{ هوَ } του αορίστου έχουν \ar{ا} στην μέση και
που στον ενεστώτα μετρέπεται είτε σε \ar{ا}, είτε σε \ar{و} είτε σε \ar{ي}.
Είναι διπλόθεμα στον αόριστο.


\begin{enumerate}
\item παίρνουμε το θέμα 3ου προσώπου \ar{ كان }
\item αφαιρούμε το \ar{ ا } από την μέση $\rightarrow$ \ar{ كـــن }
\item βάζουμε ανάμεσά τους \textit{κάποιο} τόνο και προσθέτουμε τις
καταλήξεις του αορίστου:

	\begin{enumerate}
	\item Αν το \ar{ ا } γίνεται \ar{و} τότε ο τόνος είναι \ar{ ــُـ }
	\item Αλλιώς (αν το \ar{ ا } γίνεται \ar{ي} ή \ar{ا}) τότε ο τόνος είναι \ar{ ــِـ }
	\end{enumerate}

\end{enumerate}

\begin{center}
\begin{tabular}{ c c c c }
1o Αόριστου     & $\leftarrow$ & Ενεστώτας & Αόριστος \\
\ar{ انا قُمتُ }  & $\leftarrow$ & \ar{ يقومُ } & \ar{ قامَ } \\
\ar{ انا نِمتُ }  & $\leftarrow$ & \ar{ ينامُ } & \ar{ نامَ } \\
\ar{ انا عِشتُ }  & $\leftarrow$ & \ar{ يعيشُ } & \ar{ عاشَ } \\
%\ar{ انا سِرتُ }   & $\leftarrow$ & \ar{ يسيرُ } & \ar{ سارَ } \\
%\ar{  }   & $\leftarrow$ & \ar{  } & \ar{  } \\
\end{tabular}
\end{center}
