\section*{Μικρά}
TODO

κτητ + θυλ λέξεις

Σε κάθε πρόταση που δεν βρίσκω το ρήμα πρέπει να βρώ που είναι το "είναι".

Όποια λέξη είναι στον πληθυντικό και είναι άψυχη θεωρείται γένους θυλικού:
\begin{center}
\begin{tabular}{ c c }
\ar{ الكِتاب كِبير } & το βιβλίο είναι μεγάλο  \\
\ar{ الكُتُب كِبيرة } & τα βιβλία είναι μεγάλα  \\
\ar{ هَذا كِتاب }    & αυτό είναι ένα βιβλίο \\
\ar{ هَذِهِ كُتُب }     & αυτά είναι βιβλία \\
\end{tabular}
\end{center}

Ηλιακά γράμματα

\ar{هُناك} = εκεί (there). Αν υπονοείται το "είναι" τότε \ar{هُناك} = υπάρχει (there is).

όσο γιαυτόν...

όταν μια λέξη τελειώνει σε \ar{ ا } (α) 99\% είναι \ar{ اً } (αν)

Τα επίθετα δεν παίρνουν ποτέ κτητική.

Ερώτηση + \ar{ كَم }: η λέξη μετά μπαίνει στον ενικό.
\begin{center}
\begin{tabular}{ c c }
πόσες μέρες μελετάς;  & \ar{ كَم يوم تَذاكِرُ؟ } \\
\end{tabular}
\end{center}

Σπουδάζω Χ (ιατρική, νομική, κτλ): να λέω σπουδάζω στην "σχολή" της Χ
\begin{center}
\begin{tabular}{ c c }
σπουδάζω στη νομική    & \ar{ ادرُسُ في كُلية الحُقوق } \\
σπουδάζω στην ιατρική  & \ar{ ادرُسُ في كُلية الطِب } \\
\end{tabular}
\end{center}

\section*{Μερικοί - \ar{ بعض }}
\ar{ بعض } + λέξη στον πληθυντικό - ορισμένη
\begin{center}
\begin{tabular}{ c c }
μερικοί φοιτητές              & \ar{ بعض الطُلاب } \\
μερικοί φοιτητές μου          & \ar{ بعض طُلابي } \\
μερικοί φοιτητές του παν/μίου & \ar{ بعض طُلاب الجامِعة } \\
\end{tabular}
\end{center}

\section*{Μαζί - \ar{ مع }}

\begin{center}
\begin{tabular}{ l l c }
μαζί με τον/την   & πηγαίνουμε μαζί με τη Νίοβη & \ar{ نحنُ نَذهَبُ مع نيوبي } \\
μαζί μου/σου/κτλ  & πηγαίνουμε μαζί σου         & \ar{ نحنُ نَذهَبُ معكَ } \\
μαζί (σκέτο)      & πηγαίνουμε μαζί             & \ar{ نحنُ نَذهَبُ معاً } \\
\end{tabular}
\end{center}
