\section*{Συντακτικό -  \ar{الاِعراب}}

\begin{center}
\begin{tabular}{ c c c }
Ρήμα                       & \ar{ فِعل }        & όχι \\
Υποκείμενο                 & \ar{ فاعِل }       & \ar{ـُ}\\
Αντικείμενο                & \ar{ مَفعول بهِ }   & \ar{ـَ}\\
Πρόθεση                    & \ar{ حَرف جَر }     & όχι \\
Εμπρόθετος Προσδιορισμός   & \ar{ اِسم مَجرور }  & \ar{ـِ}\\
Γενική Κτητική             & \ar{ اِضافة }      & \ar{ـِ}\\
Επίθετο                    & \ar{ صِفة }        & ότι και το ουσιαστικό του\\
\end{tabular}
\end{center}

\begin{itemize}

\item Όταν μια λέξη είναι ορισμένη παίρνει μονό τόνο.

\item Όταν μια λέξη είναι αόριστη παίρνει διπλό τόνο στο τελευταίο γράμμα.

\end{itemize}

\section*{Διπλοί τόνοι}
\begin{center}
\begin{tabular}{ c c }
\ar{ـً} & "αν"  \\
\ar{ـٍ} & "ιν"  \\
\ar{ـٌ} $\leftarrow$ \ar{ـُــُ}  & "ον"  \\
\end{tabular}
\end{center}
