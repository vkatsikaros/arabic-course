\section*{Κόλπο για να ξεχωρίζω τα ανώμαλα κούφια/μακσούρα}

Ένα πρακτικό κόλπο για να ξεχωρίζω τα ανώμαλα κούφια και τα ανώμαλα μακσούρα, είναι να κοιτώ στον Αόριστο σε ποιό πρόσωπο έχουν 2 γράμματα:

\begin{center}
\begin{tabular}{ c c c P{2.2cm} c c }
       & \multicolumn{2}{c}{Αόριστος Kούφια}    & Πού έχει 2 γράμματα;  & \multicolumn{2}{c}{Αόριστος Mακσούρα} \\
       & \multicolumn{2}{c}{\ar{ زالَ }}          &  & \multicolumn{2}{c}{\ar{ مَضى }} \\
εγώ    & \ar{ زِلتُ }    & \ar{ انا }  & $\leftarrow$   K \hspace{2.1em}  & \ar{ مَضَيتُ }  & \ar{ انا } \\
εσύ m  & \ar{ زِلتَ }    & \ar{ انتَ }  & $\leftarrow$   K \hspace{2.1em}  & \ar{ مَضَيتَ }  & \ar{ انتَ }\\
εσύ f  & \ar{ زِلتِ }    & \ar{ انتِ }  & $\leftarrow$   K \hspace{2.1em}  & \ar{ مَضَيتِ }  & \ar{ انتِ }\\
αυτός  & \ar{ زالَ }    & \ar{ هوَ }   & \hspace{2.1em} M $\rightarrow$   & \ar{ مَضى }   & \ar{ هوَ } \\
αυτή   & \ar{ زالَت }   & \ar{ هيَ }   & \hspace{2.1em} M $\rightarrow$   & \ar{ مَضَت }   & \ar{ هيَ }\\
εμείς  & \ar{ زِلنا }   & \ar{ نَحنُ }  & $\leftarrow$   K \hspace{2.1em}  & \ar{ مَضَينا } & \ar{ نَحنُ }\\
εσείς  & \ar{ زِلتُم }   & \ar{ انتُم } & $\leftarrow$   K \hspace{2.1em}  & \ar{ مَضَيتُم } & \ar{ انتُم }\\
αυτοί  & \ar{ زالوا }  & \ar{ هُم }   & \hspace{2.1em} M $\rightarrow$   & \ar{ مَضَوا }  & \ar{ هُم }\\
\end{tabular}
\end{center}
